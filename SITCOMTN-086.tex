\documentclass[SE,authoryear,toc]{article}
%\input{meta}

% Package imports go here.

% Local commands go here.

%If you want glossaries
%\input{aglossary.tex}
%\makeglossaries

\title{Rubin Baseline Calibration Plan}

% Optional subtitle
% \setDocSubtitle{A subtitle}

\author{%
Parker Fagrelius, Eli Rykoff
}

\setDocRef{SITCOMTN-086}
\setDocUpstreamLocation{\url{https://github.com/lsst-sitcom/sitcomtn-086}}

\date{\vcsDate}

% Optional: name of the document's curator
% \setDocCurator{The Curator of this Document}

\setDocAbstract{%
Current baseline plan for the year 1 minimum viable product as delivered for ORR.
}

% Change history defined here.
% Order: oldest first.
% Fields: VERSION, DATE, DESCRIPTION, OWNER NAME.
% See LPM-51 for version number policy.
\setDocChangeRecord{%
  \addtohist{1}{YYYY-MM-DD}{Unreleased.}{Parker Fagrelius}
}


\begin{document}

% Create the title page.
\maketitle
% Frequently for a technote we do not want a title page  uncomment this to remove the title page and changelog.
% use \mkshorttitle to remove the extra pages

% ADD CONTENT HERE
% You can also use the \input command to include several content files.

\appendix

\section{Introduction}

\section{Requirements}
The requirements on calibration are defined in the Science Requirements Document (LPM-17). The photometric quality and accuracy of the LSST data products is driven by four main components:
\begin{enumerate}
    \item Relative photometry (repeatability)
    \item Stability across the sky (spatial uniformity)
    \item Relative accuracy (color zero-points)
    \item Transfer to physical flux scale (external absolute photometry)
\end{enumerate}

The requirements for photometric calibration accuracy are specified using the following error decomposition (valid in the limit of small errors):
\begin{equation}
    m_{cat} = m_{true} +\sigma+\delta_{m} (x,y,\theta,\alpha,\delta,SED,t)+\Delta\,m
\end{equation}

where $m_{true}$ is the true magnitude defined by eqs. 4 and 7, $m_{cat}$ is the cataloged LSST magnitude, $\sigma$ is the random photometric error (including random calibration errors and count extraction errors), and $\Delta$m is the overall (constant) offset of the internal survey system from a perfect AB system (the six values of ∆m are equal for all the cataloged objects). 
Here, $\delta$m de- scribes the various systematic dependencies of the internal zeropoint error around ∆m, such as position in the field of view (x, y), the normalized system response ($\theta$), position on the sky ($\alpha$,$\delta$),and the source spectral energy distribution(SED).
Note that the average of $\delta$m over the cataloged area is 0 by construction.

The SRD allocates error specifications for the griz bands, with a 50\% increase expected for u and y bands. These high level allocations are further broken down to three main elements in Observatory System Specifications (LSE-30). These are Instrument Throughput, Atmospheric Transmittance, and Reference Star Catalogs. The full functional error budget can be found in LSST Document-9553.

\begin{table}[|||]
| Design Spec (millimag) | Repeatability | Uniformity | Color Accuracy | External Absolute Photometry |
| Overall Specification | 5 | | 10 | 5 | 10 |
| Instrument Throughput | 3 | 2 | 3 | - |
| Atmospheric Transmittance | 3.5 | 4 | 3 | - |
| Reference Catalog | 2.5 | 9 | 3 | - | 
\end{table}

From these functional requirements, requirements are allocated to the Telescope \& Site (LSE-60) and Data Management (LSE-61) systems to ensure that the functional requirements are met. 

\section{Calibration Approach}
\subsection{ISR Operation}
\subsection{Applying Flats}
\subsection{Global Calibration}

\section{Flat Field System}

\section{CBP}

\section{AuxTel/LATISS}

\section{Camera EOTest Data}

\section{On-Sky Data}
\subsection{Twilight Flats}
\subsection{Dense Dithered Star Fields}

\section{Plans Post-Year 1}
\subsection{Synethetic SED matched flats}
\subsection{Full CBP dataset}
% Include all the relevant bib files.
% https://lsst-texmf.lsst.io/lsstdoc.html#bibliographies
\section{References} \label{sec:bib}
\renewcommand{\refname}{} % Suppress default Bibliography section
\bibliography{local,lsst,lsst-dm,refs_ads,refs,books}

% Make sure lsst-texmf/bin/generateAcronyms.py is in your path
\appendix
\section{Calibration Products List}

\section{Acronyms} \label{sec:acronyms}
\addtocounter{table}{-1}
\begin{longtable}{p{0.145\textwidth}p{0.8\textwidth}}\hline
\textbf{Acronym} & \textbf{Description}  \\\hline

CBP & Collimated Beam Projector \\\hline
GPS & Global Positioning System \\\hline
ISR & Instrument Signal Removal \\\hline
LATISS & LSST Atmospheric Transmission Imager and Slitless Spectrograph \\\hline
LPM & LSST Project Management (Document Handle) \\\hline
LSE & LSST Systems Engineering (Document Handle) \\\hline
LSST & Legacy Survey of Space and Time (formerly Large Synoptic Survey Telescope) \\\hline
LTS & LSST Telescope and Site  (Document Handle) \\\hline
NIST & National Institute of Standards and Technology (USA) \\\hline
QE & quantum efficiency \\\hline
SE & System Engineering \\\hline
SED & Spectral Energy Distribution \\\hline
SF & Structure Function \\\hline
SRD & LSST Science Requirements; LPM-17 \\\hline
\end{longtable}

% If you want glossary uncomment below -- comment out the two lines above
%\printglossaries





\end{document}
